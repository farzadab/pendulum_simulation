\documentclass[a4paper,11pt]{article}
\usepackage[T1]{fontenc}
\usepackage[utf8]{inputenc}
\usepackage{lmodern}
\usepackage{amsmath}
\usepackage[colorlinks, allcolors=blue]{hyperref}
\usepackage{graphicx}
\usepackage{array}

\newcolumntype{C}{>{\centering\arraybackslash}m{2.2cm}}
\newcolumntype{D}{>{\arraybackslash}m{.4cm}}
\newcolumntype{E}{>{\centering\arraybackslash}m{2.13cm}}
\newcolumntype{N}{@{}m{0pt}@{}}

\title{Assignment 4}
\author{Farzad Abdolhosseini}

\begin{document}

\maketitle

\section{One-link pendulum}
First, we write out the equations of motion:
\begin{enumerate}
  \item Newton equation:
    \begin{align*}
      \sum F &= M_1 \ddot{P_1}\\
      \Rightarrow F_c + M_1 g &= M_1 \ddot{P_1}\\
      \Rightarrow M_1 \ddot{P_1} - F_c &= M_1 g
    \end{align*}
    
  \item Euler equation:
    \begin{align*}
      \sum \tau &= I_{1_w} \dot{\omega_1} + \omega_1 \times I_{1_w} \omega_1\\
      \Rightarrow r_w \times F_c &= I_{1_w} \dot{\omega_1} + \omega_1 \times I_{1_w} \omega_1\\
      \Rightarrow I_{1_w} \dot{\omega_1} - r_w \times F_c &= \omega_1 \times I_{1_w} \omega_1\\
 I_{1_w} \dot{\omega_1} - \widetilde{r_w} F_c &= \omega_1 \times I_{1_w} \omega_1\\
    \end{align*}
    
  \item Fixed point constraint:
    \begin{align*}
      &P_{1_c} = A\\
      \Rightarrow\quad &\dot{P_{1_c}} = 0\\
      \Rightarrow\quad &\ddot{P_{1_c}} = 0\\
      \Rightarrow\quad &\ddot{P_{1_c}} = \ddot{P_1} + \dot{\omega_1} \times r_w + \omega_1 \times (\omega_1 \times r_w) = 0\\
      \Rightarrow\quad &-\ddot{P_1} + \widetilde{r_w}\dot{\omega_1} = \omega_1 \times (\omega_1 \times r_w)\\
    \end{align*}
\end{enumerate}


Now, we have to arrange them in a matrix in the form of $Ax = b$:
\begin{tabular}{|C|C|C|N}
  \hline
  M & & $\begin{array}{DDD} -1 &  &  \\[.4cm]  & -1 &  \\[.4cm]  &  & -1 \\[.4cm] \end{array}$ & \\[2.2cm]
  \hline
   & I & $-\widetilde{r_w}$ & \\[2.2cm]
  \hline
  $\begin{array}{DDD} -1 &  &  \\[.4cm]  & -1 &  \\[.4cm]  &  & -1 \\[.4cm] \end{array}$ & $\widetilde{r_w}$ & & \\[2.2cm]
  \hline
\end{tabular}
\begin{tabular}{|D|N}
  \hline
  $\ddot{P_1}$ &\\[2.2cm]
  \hline
  $\dot{\omega_1}$ &\\[2.2cm]
  \hline
  $F_c$ &\\[2.2cm]
  \hline
\end{tabular}
=
\begin{tabular}{|E|N}
  \hline
  $M_1 g$ &\\[2.2cm]
  \hline
  $\omega_1 \times I_{1_w} \omega_1$ &\\[2.2cm]
  \hline
  $\omega_1 \times (\omega_1 \times r_w)$ &\\[2.2cm]
  \hline
\end{tabular}

\vspace{6mm}
Then, the only missing link is to calculate the moment of inertia. By symmetry, $I_{1_L}$, i.e. innertia in the local frame, must be a diagonal matrix. Also, a \href{https://en.wikipedia.org/wiki/List_of_moments_of_inertia}{quick search} gives the formula for a solid cuboid as:
\begin{align*}
  I_{1_l} = \frac{m}{12}\begin{bmatrix}w^2 + d^2 & 0 & 0\\ 0 & d^2 + h^2 & 0\\ 0 & 0 & w^2+h^2\end{bmatrix}
\end{align*}

Therefore, we can get the innertia in the world frame just by applying the rotation matrix:
$$I_{1_w} = R \; I_{1_l} \; R^{-1}$$

Also, it's important to note that the $r_w$ here refers to the distance from the center of mass of the pendulum to the fixed point in the world frame which can be calculated again using the rotation matrix:
$$ r_w = R \;r_l $$

\end{document}
